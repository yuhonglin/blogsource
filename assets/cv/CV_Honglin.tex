%%
%% The following code sets up the document formatting
%%

%this assumes that res_yy.sty is in some path
\documentstyle[hyperref, margin, line]{res_yy}

\hypersetup{backref,pdfpagemode=Full,colorlinks=False,backref,hidelinks}

\addtolength{\oddsidemargin}{-0.45in}
\addtolength{\voffset}{-0.30in}
\addtolength{\textwidth}{1.00in} \addtolength{\textheight}{1.50in}

\renewcommand{\namefont}{\LARGE\emph}

% \newcommand{\hl}[1]{{#1}}
\newcommand{\hl}[1]{{\bf #1}}


%%
%% The following code defines some macros for terms which have raised font
%% (ie 4\fourth would result 4th with the 'th' raised (superscripted)
%%

\def\Cplusplus{{\rm C\raise.5ex\hbox{\small ++}}}
\def\CSharp{{\rm C\raise.5ex\hbox{\small \#}}}
% 'st' 'nd' 'rd' 'th' superscripts for numbers
\def\first{{\raise.5ex\hbox{\small st}}}
\def\second{{\raise.5ex\hbox{\small nd}}}
\def\third{{\raise.5ex\hbox{\small rd}}}
\def\fourth{{\raise.5ex\hbox{\small th}}}

% Simpler bibsection for CV sections              
% (thanks to natbib for inspiration)
\makeatletter
\newlength{\bibhang}
\setlength{\bibhang}{1em} %1em}
\newlength{\bibsep}
 {\@listi \global\bibsep\itemsep \global\advance\bibsep by\parsep}
\newenvironment{bibsection}%
        {\begin{enumerate}{}{%
%        {\begin{list}{}{%
       \setlength{\leftmargin}{\bibhang}%
       \setlength{\itemindent}{-\leftmargin}%
       \setlength{\itemsep}{\bibsep}%
       \setlength{\parsep}{\z@}%
        \setlength{\partopsep}{0pt}%
        \setlength{\topsep}{0pt}}}
        {\end{enumerate}\vspace{-.6\baselineskip}}
%        {\end{list}\vspace{-.6\baselineskip}}
\makeatother

%%
%% starting the actual document
%%


\begin{document}

%the name in big fonts at the top of resume
%this is left aligned
\name{Honglin Yu}

%this is right aligned
\address{
School of Computer Science, Australian National University, NICTA  \hspace{2.4cm} Email: honglin.yu@anu.edu.au 
}
\address{
Homepage : \href{http://yuhonglin.github.io}{http://yuhonglin.github.io}
\ \ \ \ \ \ \ \ \ \ \ \ \ \ \ \ \ \ \ \ \ \  \ \ \ \hspace{4cm}   Mobile: +61 406502886
}

%\section{Contact Information}

% NOTE: Mind where the & separators and \\ breaks are in the following
%       table.
%
% ALSO: \rcollength is the width of the right column of the table
%       (adjust it to your liking; default is 1.85in).
%
%\newlength{\rcollength}\setlength{\rcollength}{1.4in}%
%
%\begin{tabular}[t]{@{}p{\textwidth-\rcollength}p{\rcollength}}
%\href{http://www.cse.osu.edu/}%
%     {Department of Computer Science and Engineering} & \\
%\href{http://www.osu.edu/}{The Ohio State University}
%8151 33rd Ave S Unit 409   & 507-304-1668 \\
%Bloomington, MN  55425     & \email{quic0038@umn.edu}\\
%\end{tabular}

\begin{resume}



%%
%% This section of code is inelegant, but I'm too lazy to fix it
%%

\section{\textsc{Research  Interests}}
\ \ \ Data Mining, Machine Learning, Online Social Network Analysis
%Stochastic Investment Models, Stochastic Mortality Models, Pensions, Derivatives Pricing, Risk Management

\section{\textsc{Education}}

\textbf{Australian National University}  \ \  \ \ \ \ \ \ \ \ \ \ \ \ \ \ \ \ \ \ \ \ \ \ \ \ \ \ \ \ \ \ \ \ \ \ \ \ \ \ \ \ \ \ \ \ \ \ \ \ \ \ \ \   Aug. 2011 - present \\
\emph{Ph.D. in Computer Science} 


\begin{itemize}
  \item Thesis Topic: Popularity Analysis of YouTube Videos
    %   % \item Supervisors: Adam Butt (Chair), Bridget Browne, Tim Higgins, Kin-Yip Ho, Stuart Leckie
  \item Work at Machine Learning group of NICTA
  \item Thesis will be submitted before Aug. 2015
  %\item Supervisors: Adam Butt(Chair), Bridget Browne, Tim Higgins, Kin-Yip Ho, Stuart Leckie 
 \end{itemize}

\textbf{Southeast University, China} \ \ \ \ \ \ \ \ \ \ \ \ \ \ \ \ \ \ \ \ \ \ \ \ \ \ \ \ \ \ \ \ \ \ \ \ \ \ \ \ \ \ \ \ \ \ \ \ \  \ \ \ \   Aug. 2009 - Jul. 2011 \\ 
\emph{Finished master course in Automation}

\textbf{Southeast University, China} \ \ \ \ \ \ \ \ \ \ \ \ \ \ \ \  \ \ \ \ \ \ \ \ \ \ \ \ \ \ \ \ \ \ \ \ \ \ \ \ \ \ \ \ \ \ \  \ \ \ \ \ \ \ \ \ \ \ \ Aug. 2005 - June 2009 \\ 
\emph{Bachelor of Automation} 

 \begin{itemize}
   \item Rank: 4/140
   \item Best Graduation Thesis (2 out of 140)
 \end{itemize}


% \section{\textsc{Professional Designation}}
% Working to complete the requirements for an \textbf{Associate of the Society of Actuaries (ASA)} \\
% % Date expected to attain the designation: 2015 
% Passed SOA Exams: P, FM, MFE 

% \textbf{Chartered Enterprise Risk Analyst (Candidate)} 
% \textbf{National Computer Rank Examination Certificate} \ \ \ \ \ \ \ \ \ \ \ \ \ \ \ \  \ \ \ \ \ \ \ \ \ \ \ \ \ \ \ \ \ \ \ \ \ \ \ \ \ \ \ \ \ \ \  \ \ \ \ \ \ \ \ \ 2007 \\ 
% Computer Language: C 





%%
%% the meat of the resume starts now
%%

% \begin{formatb}
%   \employer{} \dates{r}\\
%   %\title{r}
%   \location{l} \\
%   \body\\
% \end{formatb}

\section{Skill Highlights}
%\begin{bibsection} 
\begin{itemize}
\item Experienced in \hl{Data Science} and \hl{Statistical Analysis}
  \begin{itemize}
  \item Classification, regression, clustering, time series analysis
  \item Data exploration and visualization (matplotlib, R, d3.js)
  \item Working on large scale \hl{real data} for $>$3 years
  \end{itemize}
\item Broad knowledge in \hl{Machine Learning}
  \begin{itemize}
  \item Deep understanding of main-stream algorithms (SVM, Random Forest, ANN etc.)
  \item Experienced in carrying out experiments and \hl{feature engineering}
  \end{itemize}
\item Firm programming skills, especially in \hl{C/C++} and \hl{python}
  \begin{itemize}
    \item Also familiar with SQL, Javascript, R, Julia
    \item Use Linux everyday for more than 6 years, skilled emacs user
  \end{itemize}      
\item Experienced in collecting and analyzing large scale Internet data
  \begin{itemize}
    \item Skilled in implementing various online data \hl{crawlers}
    \end{itemize}
\item Domain knowledge related to Computational Social Science Research
  \begin{itemize}
  \item YouTube video popularity analysis
  \item Twitter user network analysis
  \end{itemize}
\item  Basic controller design
\end{itemize}
%\end{bibsection}


\section{\textsc{Research Projects}}
% \begin{bibsection}
\ \ \ \  Please see \href{http://yuhonglin.github.io/project/}{http://yuhonglin.github.io/project/} for more details \\
\begin{itemize}
%(90\% contribution: idea, data, methodology, and writing)
\item \href{http://yuhonglin.github.io/2013/10/02/youtube-and-twitter.html}{Analyzing Twitter-driven YouTube Views}
  \begin{itemize}
  \item Predicting videos' viewcount with Twitter's feed (\href{http://cantabile.anu.edu.au/yt/demo/}{\bf Demo})
  \item Contributions: Most of the ideas; All the Software implementation, data collection, analysis and prediction
  \end{itemize}
\item \href{http://yuhonglin.github.io/2014/08/31/visual-meme-sequence-analysis.html}{Data Mining on YouTube Video Shot Sequences}
  \begin{itemize}
  \item Mining frequent remixed shots from large number of videos
  \item Contributions: Part of the ideas and data collection; All the implementation, data preprocessing and analysis
  \end{itemize}
\item \href{http://yuhonglin.github.io/2012/10/04/viewcount-phase.html}{Exploring the Phases of Popularity Evolution of YouTube Videos}
  \begin{itemize}
  \item Understanding viewcount dynamics by novel time series segmentation techniques
  \item Contributions: Most of the ideas; All the software implementation, data collection, analysis and prediction
  \end{itemize}    
\end{itemize}
%\end{bibsection}

\begin{formatb}
  \employer{l}\dates{r}\\
  \body\\
\end{formatb}

\section{\textsc{Software Highlights}}
\begin{itemize}
\item \href{https://github.com/yuhonglin/YTCrawl}{YTcrawl}: a YouTube video history viewcount crawler
\item SegFit : a time series segmentation algorithm written in C++
\item \href{https://github.com/yuhonglin/shotdetect}{Shotdetect} : a video shot detection program written in C++
\end{itemize}

\section{\textsc{Media Coverage}}
\begin{itemize}
  \item {\bf ANU Reporter} \ \ {\it How the viral video star is born}. \ August 2015.\\
    \url{http://www.anu.edu.au/news/all-news/how-the-viral-video-star-is-born}
  \item {\bf NCI Research News} \ \ {\it Predicting popularity}. \ September 2015.\\
    \url{http://nci.org.au/2015/09/30/predicting-popularity/}
\end{itemize}


\section{Papers}
\begin{itemize}
\item {\bf Honglin Yu}, Lexing Xie and Scott Sanner, Exploring the Popularity Phases of YouTube Videos: Observations, Insights, and Prediction. In proceeding of International AAAI Conference on Web and Social Media (ICWSM) 2015.
\item {\bf Honglin Yu}, Lexing Xie and Scott Sanner, Twitter-driven YouTube Views: Beyond Individual Influencers. In proceeding of ACM Multimedia Conference (ACMM) 2014.  
\end{itemize} 



\section{\textsc{Teaching Experience}}
\begin{itemize}
  \item Computational Social Science Summer Short Course \ \ \ \ \ \ \ \ \  \ \ \ \hspace{3cm} Jul. 2013 \\
    Teaching Assistant, Beihang University, China
  \item Computer Network \ \ \ \ \ \ \ \ \ \ \ \ \ \hspace{9cm} 2011 \\
    Teaching Assistant, Southeast University, China
\end{itemize} 


\section{\textsc{Awards}}
\begin{itemize}
% \item Postgraduate Scholarship of China Scholarship Council (CSC)  \hfill 2011
\item Excellent Graduation Dissertation of Southeast University (2 out of 140) \hfill  2009
\item Champion of the 3D Soccer Simulation League, Robocup Worldcup \hfill 2008
\item National Mathematical Modeling Contest Third Prize of Jiangsu Province \hfill  2008
\item Higher Mathematics Competition of Jiangsu Province, Third Prize \hfill 2006
\end{itemize}

\begin{formatb}
  \employer{l}\dates{r}\\
  \body\\
\end{formatb}

\section{\textsc{Review Experience}}
\begin{itemize}
  \item International World Wide Web Conference (WWW)
  \item ACM International Conference on Web Search and Data Mining (WSDM)
\end{itemize}


\section{\textsc{Language}}
Chinese (native) \hspace{3cm} English (fluent)



%  \section{\textsc{References}}
% \begin{ncolumn}{2}
% Dr. Lexing Xie & Dr. Scott Sanner\\
% Senior Lecturer  &  Assistant Professor  \\   
% Research School of Computer Science & Computer Science  \\                                      
% Australian National University & College of Engineering \\
% affiliated with NICTA & Oregon State University \\
% Email: lexing.xie@anu.edu.au & Email: scott.sanner@oregonstate.edu \\
% Phone: (+61) 2 612 51646 & \\
% \\

% Dr. Henry Gardner \\ %& Dr. Kin-Yip, Ho \\    
% Associate Professor and Reader \\
% College of Engineering \& Computer Science \\ %& College of Business and Economics\\
% Australian National University \\ % & Australian National University\\
% Email:  henry.gardner@anu.edu.au\\ % & Email: kin-yip.ho@anu.edu.au\\
% Phone: (+61) 2 612 58181 \\ %& Phone: (+61) 2 612 57299


% \end{ncolumn}


\end{resume}
\end{document}

%%% Local Variables:
%%% mode: latex
%%% TeX-master: t
%%% End:
